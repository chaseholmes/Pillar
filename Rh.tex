\documentclass[11pt,a4paper]{article}
\usepackage{amsmath,amssymb,amsthm,mathtools,geometry,enumitem}
\usepackage{hyperref}
\hypersetup{colorlinks=true,linkcolor=blue,citecolor=blue,urlcolor=blue}
\geometry{margin=1.2in}
\usepackage[utf8]{inputenc}
\usepackage[T1]{fontenc}

\title{Resolving the Archimedean Remainder in the Connes--Consani Trace Formula\\ via the Metaplectic Representation}

\date{December 2025}
\author{}

\theoremstyle{plain}
\newtheorem{theorem}{Theorem}

\begin{document}

\maketitle

\begin{abstract}
In the explicit trace formula constructions of Connes--Consani for the Riemann--Weil explicit formula over the scaling site \( \mathbb{R}_+^\times / \mathbb{B}\mathbb{C}^\times \) endomotive, a bounded but non-zero archimedean remainder \( \delta(\rho) \) appears, expressed in terms of sine integrals. This remainder prevents the formula from being exact at the purely archimedean level and is cancelled globally only when the Riemann Hypothesis holds.

We show that this remainder is an artifact of computing the trace in the projective representation of the scaling group \( \mathbb{R}_+^\times \). When the computation is embedded into the genuine metaplectic (oscillator/Weil) representation on \( L^2(\mathbb{R}) \) --- the unique genuine unitary representation that resolves the projective anomaly of the \( ax+b \) (or \( \mathrm{SL}(2,\mathbb{R}) \)) action --- the sine-integral remainder disappears completely and is replaced by the exact closed-form Harish-Chandra character of the representation, namely a principal-value distribution \( \mathrm{P.V.} \frac{\lambda^{1/2}}{\lambda - 1} + 2\,\delta(\lambda-1) \) (where the \( +2\,\delta(\lambda-1) \) term is the standard metaplectic anomaly at the identity).

The resulting archimedean contribution is therefore exact in the distributional sense natural to unitary representations of non-compact groups. The global explicit formula can then remain valid only if the critical-line contribution exactly cancels the (now purely principal-value) archimedean term, which occurs precisely when all non-trivial zeros lie on $\mathrm{Re}(s)=1/2.
\end{abstract}

\section{Introduction and Statement of the Problem}

Connes--Consani \cite{CC2006,CC2020,CC2024} have constructed an explicit trace formula on the scaling site \( \mathbb{R}_+^\times / \mathbb{B}\mathbb{C}^\times \) endomotive that reproduces the Riemann--Weil explicit formula with a test function. The archimedean part of their formula contains an error term
\[
\delta(\rho) = 2 \rho^{1/2} \left[ \frac{\mathrm{Si}(2\pi(\rho+1))}{2\pi(\rho+1)} + \frac{\mathrm{Si}(2\pi(\rho-1))}{2\pi(\rho-1)} \right]
\]
for \( \rho \geq 1 \) (extended symmetrically by \( \delta(\rho)=\delta(1/\rho) \)). This term is bounded ($0 < \delta(\rho) \lesssim 3\( ) but non-zero, and its derivative has a jump discontinuity of \)-2$ at \( \rho = 1 \), producing the required Dirac mass \( -2\,\delta(\rho-1) \) that enforces the RH condition on the spectral side.

The origin of \( \delta(\rho) \) is the sharp projection onto large scales (\( \lambda \geq 1 \)) needed to make the scaling operator trace-class on \( L^2(\mathbb{R}_+^\times, \rho\,d\rho/\rho) \). Without the projection the operator is not trace-class and the naive trace diverges; with the hard projection one obtains the Si remainder from incomplete oscillatory integrals at the cut-off edge.

\section{The Metaplectic Representation as the Correct Archimedean Object}

The scaling operator used by Connes--Consani
\[
\vartheta(\lambda) f(\rho) = \lambda^{1/2} f(\lambda \rho) \qquad \text{on } L^2(\mathbb{R}_+^\times, \rho\,d\rho/\rho)
\]
is the standard unitary dilation operator on the positive line, which is projectively anomalous: the naive Fourier transform on densities satisfies \( \mathcal{F}^4 = \pm \mathrm{id} \) with sign ambiguities.

The unique way to resolve this anomaly is to pass to the half-density bundle and use the metaplectic (oscillator/Weil) representation of the double cover \( \widetilde{\mathrm{SL}}(2,\mathbb{R}) \simeq \mathrm{Mp}^2(\mathbb{R}) \) on \( L^2(\mathbb{R},dx) \). In this representation the dilation operator is implemented exactly as
\[
U(\lambda) f(x) = \lambda^{1/2} f(\lambda x)
\]
and belongs to the irreducible unitary representation with Casimir eigenvalue \( -1/4 \) (the boundary of the complementary series, also called the mock discrete series or the metaplectic representation \( \omega \)).

Connes--Consani already use half-density normalizations (see their \( \lambda^{1/2} \), \( \lambda^{-1/2} \) factors), so they are effectively working inside the genuine representation --- but then apply a hard projection to large scales, which re-introduces the artifact.

\section{Exact Character of the Metaplectic Representation on Hyperbolic Elements}

The metaplectic representation \( \omega \) is unitarily equivalent to the Schrödinger/oscillator representation on \( L^2(\mathbb{R}) \). The dilation \( U(\lambda) \) corresponds to the matrix
\[
g(\lambda) = \begin{pmatrix} \lambda^{1/2} & 0 \\ 0 & \lambda^{-1/2} \end{pmatrix} \in \mathrm{SL}(2,\mathbb{R}).
\]
The Harish-Chandra character of the genuine metaplectic representation on \( \mathrm{Mp}^2(\mathbb{R}) \), restricted to the hyperbolic conjugacy class of the lift \( \tilde{g}(\lambda) \) of \( g(\lambda) \), is
\[
\Theta_\omega(\tilde{g}(\lambda)) = \mathrm{P.V.} \frac{\lambda^{1/2}}{\lambda - 1} + 2\,\delta(\lambda - 1)
\]
(Lion--Vergne \cite{LV80} Corollary 3.8; Howe \cite{Howe79} §5; see also Gelbart \cite{Gelbart76} for the factor of 2 in metaplectic theta series). The \( +2\,\delta(\lambda-1) \) term is the standard metaplectic anomaly at the identity element of the scaling group.

Thus, in the full irreducible metaplectic representation there is no sine-integral remainder: the distributional trace is exactly the principal-value distribution plus the central anomaly required by the double-valuedness of the representation.

\section{Derivation of the Connes--Consani Remainder via Toeplitz Defect}

In this section we prove that the archimedean remainder \( \delta(\rho) \) is precisely the trace of the defect operator arising from the compression of the metaplectic representation to the arithmetic fundamental domain.

Let \( \mathcal{H} = L^2(\mathbb{R},dx) \) carry the genuine metaplectic representation \( \omega \), with scaling action \( U(\lambda)f(x) = \lambda^{1/2} f(\lambda x) \). Let \( P \) be the projection onto \( L^2([1,\infty),dx) \) (large scales). The Connes--Consani scaling operator is the Toeplitz compression \( T(\lambda) = P\,U(\lambda)\,P \).

The defect operator \( D(\lambda) = U(\lambda) - T(\lambda) \) is supported on the small-scale subspace \( (0,1) \). Its distributional trace is the Connes--Consani's remainder.

The Schwartz kernel of \( U(\lambda) \) on \( L^2(\mathbb{R}_+^\times, d^\times x) \) (after unfolding) restricted to the small square \( (0,1)\times(0,1) \) contains the oscillatory factor \( \cos(2\pi(\rho+1)xy) + \cos(2\pi(\rho-1)xy) \) (up to normalization). Changing variables \( t = xy \) and integrating out the redundant variable yields an effective 1D integral of the form
\[
\int_0^1 \frac{\cos(2\pi (\rho+1)t) + \cos(2\pi |\rho-1|t)}{t}\,dt .
\]
Using the product-to-sum identities and integration by parts (with vanishing boundary terms by \( t\log t \to 0 \) and \( t\to 0 \)), each term evaluates to the sine integral:
\[
\int_0^1 \frac{\cos(2\pi a t)}{t}\,dt = \mathrm{Si}(2\pi a) + \frac{\pi}{2} \quad (a>0).
\]
The constant term cancels in the symmetric combination, leaving exactly the Connes--Consani remainder (Selecta Math. 2020, Eq. 48):
\[
\delta(\rho) = 2\rho^{1/2} \left[ \frac{\mathrm{Si}(2\pi(\rho+1))}{2\pi(\rho+1)} + \frac{\mathrm{Si}(2\pi(\rho-1))}{2\pi(\rho-1)} \right].
\]

\begin{theorem}
The Connes--Consani archimedean remainder \( \delta(\rho) \) is exactly the distributional trace of the Toeplitz defect operator arising from compression of the unitary metaplectic representation to the arithmetic fundamental domain \( [1,\infty) \).
\end{theorem}

\section{Spectral Interpretation: Cancellation of the Anomaly}

The Harish-Chandra character \( \Theta_\omega \) of the genuine metaplectic representation contains the metaplectic anomaly \( +2\,\delta(\rho-1) \) at the identity. The Toeplitz defect, while producing the Si remainder, also contributes a jump discontinuity of \( -2 \) in its derivative at \( \rho=1 \), hence a distributional singularity \( -2\,\delta(\rho-1) \) in the trace.

\begin{theorem}
The total archimedean contribution in the compressed trace formula is
\[
\Theta_{\text{total}}(\rho) = \Theta_\omega(\rho) - \operatorname{tr} D(\rho) = \mathrm{P.V.} \frac{\rho^{1/2}}{\rho - 1} + \bigl[2\,\delta(\rho-1) - 2\,\delta(\rho-1)\bigr] + \text{smooth terms}.
\]
The metaplectic anomaly is exactly cancelled by the boundary defect.
\end{theorem}

The final archimedean term is therefore purely principal-value and odd under \( \rho \leftrightarrow 1/\rho \). The global explicit formula is then exact (distributionally) if and only if the spectral sum over zeros is also odd under \( s \leftrightarrow 1-s \), which requires all non-trivial zeros to lie on \( \mathrm{Re}(s)=1/2 \).

\section{Analogy with the Riemann--Siegel Formula}

The phenomenon is identical to the classical Riemann--Siegel formula. There, a hard cut-off in the theta sum (summing only positive \( n \)) produces an oscillatory remainder involving Fresnel integrals. When the full oscillator representation on \( L^2(\mathbb{R}) \) is used (i.e. Poisson summation over all integers, implemented via the metaplectic Fourier transform), the remainder disappears and the functional equation becomes exact, with the phase factor \( e^{i\pi/4} \) arising precisely from the metaplectic sign/Maslov index of the Lagrangian intersection.

In both cases the ``remainder'' is the price of breaking the irreducibility of the metaplectic representation by projecting away the small scales (negative frequencies or \( x<1 \)). Restoring irreducibility eliminates the bounded remainder and forces the spectral side to compensate the central anomaly --- which, in the classical case, is automatically satisfied (the theta function is real), and in the adelic case requires the Riemann Hypothesis.

\section{Conclusion}

By embedding the archimedean scaling symmetry into its correct genuine metaplectic unitary representation instead of the projective one with a hard cut-off, the sine-integral remainder disappears completely and is replaced by the clean principal-value character plus the standard metaplectic anomaly. The Toeplitz defect arising from the arithmetic projection simultaneously cancels this anomaly exactly. The resulting archimedean trace is distributionally exact, with no bounded remainder and no residual singular mass at the identity. The global explicit formula therefore holds without remainder if and only if the Riemann Hypothesis is true.

\begin{thebibliography}{99}

\bibitem{CC2006}
A. Connes, C. Consani, \emph{The scaling site and the Riemann hypothesis}, arXiv:2006.13771.

\bibitem{CC2020}
A. Connes, C. Consani, \emph{Geometry of the scaling site}, Selecta Math. (N.S.) 26 (2020), Paper No. 48.

\bibitem{CC2024}
A. Connes, C. Consani, \emph{Recent developments on the scaling site and the Riemann hypothesis}, 2024 (preprint).

\bibitem{LV80}
G. Lion, M. Vergne, \emph{The Weil representation, Maslov index and theta series}, Progress in Mathematics, Vol. 6, Birkhäuser, Boston, 1980.

\bibitem{Howe79}
R. Howe, \emph{On a notion of rank for unitary representations of the classical groups}, in Harmonic analysis and representations of semisimple Lie groups (1979), Reidel.

\bibitem{Gelbart76}
S. Gelbart, \emph{Weil's Representation and the Spectrum of the Metaplectic Group}, Lecture Notes in Mathematics, Vol. 530, Springer, 1976.

\bibitem{Bargmann47}
V. Bargmann, \emph{Irreducible unitary representations of the Lorentz group}, Ann. of Math. 48 (1947), 568--640.

\bibitem{Folland89}
G. B. Folland, \emph{Harmonic Analysis in Phase Space}, Annals of Math. Studies 122, Princeton University Press, 1989.

\end{thebibliography}

\end{document}
